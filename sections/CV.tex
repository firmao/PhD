\section{Abstract}
Andre Valdestilhas is a Ph.D. candidate at the University of Leipzig. 
\todo{Need to put the CV here.}
% He worked for six years in one of the largest South America research laboratories, Tecgraf, focusing mostly on designing and developing Geographic Information Systems. During this period Edgard participated in several projects, among others in: 

% \begin{itemize}
%     \item Cooperative development of open-source libraries, such as Geotools and TerraLib in partnership with the Spatial Research Institute (INPE);
%     \item Development of systems for exclusive use of Petrobras and Liquigas as InfoPAE and InfoPAE-MOBILE;
%     \item Urban mass transit monitoring systems for M2M.
% \end{itemize}

% Since 2013 Edgard is working on Natural Language Processing (NLP) focusing on Question Answering and Semantic Search systems.
% He is an active member of the Semantic and NLP academic community participating as a reviewer in conferences and working groups such as AAAI (Conference on Artificial Intelligence) and QALD (Open Challenge on Question Answering over Linked Data) as well as publishing several peer-reviewed publications in the area.

% \section{Work experience}

% \begin{eventlist}

% \item{2011 -- 2014}
%      {Geobrainz}
%      {Co-founder \& Manager}

% - Leader;

% - Technology and trends orientation;

% - System Architect.

% \item{2012 -- 2012}
%      {Terragis}
%      {Developer Coordinator}

% Concept, architect and developer coordinator of Latitude system.

% \item{2011 -- 2011}
%      {PUC-Rio/Hewlett-Packard}
%      {Researcher}

% R3 - Formalizing and reasoning over responsibilities in the RDA methodology.
% \newline
% \newline
% The goal of R3 project was to define a framework (models, algorithms and tools) to formalize and to support automated reasoning over responsibilities and network of responsibilities that arise from the analysis of Roles in the contextual layer of the Role Domain Architecture Methodology.

% \personal
%     [aksw.org/EdgardMarx]
%     {18. Oktober, 17\newline 04103 -- Leipzig}
%     {+39 (30) 6862332}
%     {marx@informatik.uni-leipzig.de}
%     {www.linkedin.com/in/digamarx}

% \item{2005 -- 2011}
%      {TECGRAF}
%      {Researcher, System Designer \& Developer}
     
% \href{http://www.tecgraf.puc-rio.br}{http://www.tecgraf.puc-rio.br}
% \newline

% - Cooperative development of open source libraries as \href{http://geotools.org}{Geotools} and \href{http://www.terralib.org/}{TerraLib} in partnership with \href{http://www.inpe.br/}{INPE};

% - Development of monitoring and emergency planning systems for Petrobras and  Liquigas as InfoPAE and InfoPAE-MOBILE;

% - Development of monitoring system for urban mass transportation for M2M, among others.



% \end{eventlist}



% \section{Education}

% \begin{yearlist}

% \item{2012 -- Inc. }
%      {PhD. Computer Science}
%      {University of Leipzig, Germany}

% \item{2009 -- 2011}
%      {MSc. Computer Science}
%      {Pontifical Catholic University of Rio de Janeiro, Brazil}

% \item{2003 -- 2006}
%      {BSc. Computer Science}
%      {Pontifical Catholic University of Rio de Janeiro, Brazil}
     
% \item{1999 -- 2001}
%      {High School Graduation}
%      {Technical School of Administration, Brazil}

% \end{yearlist}

% \section{Projects}

% \subsection{Open source}

% \begin{yearlist}

% \item{2016 -- Pres.}
%      {KBox \newline \href{http://github.com/AKSW/kbox}{\footnotesize https://github.com/aksw/kbox}}
%      {Shifting Query Execution on Knowledge Graphs to the Edge.}

% \item{2014 -- Pres.}
%      {openQA \newline \href{http://openqa.aksw.org}{\footnotesize http://openqa.aksw.org} \href{http://bitbucket.org/emarx/openqa}{\footnotesize http://bitbucket.org/emarx/openqa}}
%      {openQA is a modular and open source framework that allows the easy development and evaluation of question answering and semantic search systems.}

% \end{yearlist}
  
% \subsection{Proprietary}

% \begin{yearlist}

% \item{2011 -- 2014}
%      {Orbitrack {\footnotesize\href{http://www.orbitrack.com.br}{http://www.orbitrack.com.br}}}
%      {A mass monitoring System based on Google Cloud Plataform designed to store and manage massive multi-sensor data.}

% \item{2012 -- 2012}
%      {Latitude {\footnotesize\href{http://www.latitude.org.br}{http://www.latitude.org.br}}}
%      {Better planning and management of public resources by identifying weaknesses and strengths in governmental programs.}

% \item{2006 -- 2009}
%      {InfoPAE (PETROBRAS) \newline {\footnotesize \href{http://www.petrobras.com.br}{http://www.petrobras.com.br}}}
%      {Contingency action plan system designed to guide workers in case of Emergency.}
     
% \item{2006 -- 2009}
%      {InfoPAE-MOBILE (LIQUIGAS) \newline {\footnotesize \href{http://www.liquigas.com.br}{http://www.liquigas.com.br}}}
%      {Mass monitoring sytem designed for monitoring the fuel transportation and distribution of over two thousand vehicles.}
     
% \item{2006 -- 2009}
%      {TDK (Terralib Development Toolkit)}
%      {Toolkit of functions designed to build geographic applications with Terralib.}

% \item{2006 -- 2009}
%      {TDK GPS Library}
%      {Library designed to deal with GPS device data implementing functions such as extraction, storage and visualization.}

% \item{2005 -- 2006}
%      {FROTA (M2M) \newline {\footnotesize \href{http://www.m2msolutions.com.br}{http://www.m2msolutions.com.br}}}
%      {Mass monitoring system designed for route planning and monitoring of public transportation.}
     
% \end{yearlist}

% \subsection{Research}
% \begin{yearlist}


% \item{2015 -- Pres.}
%      {SMART}
%      {A Semantic Search Engine.}

% \item{2015 -- Pres.}
%      {DBtrends {\footnotesize\href{http://dbtrends.aksw.org/}{http://dbtrends.aksw.org}}}
%      {A framework for Publishing and Evaluating RDF ranking functions.}
     
% \item{2013 -- Pres.}
%      {RDFSlice}
%      {A large-scale approach for RDF dataset slicing.}
         
% \item{2012 -- 2012}
%      {R3 (PUC-RIO / Hawlet Package)}
%      {Formalizing and reasoning over responsibilities in the RDA methodology.}

% \item{2011 -- 2011}
%      {RDB2RDF}
%      {A tool for generating RDB2RDF mappings.}
 
% \end{yearlist}

% \subsection{Small Contrubutions}
% \begin{yearlist}

% \item{2012 -- 2012}
%      {Geotools \newline {\footnotesize\href{http://geotools.org/}{http://geotools.org}}}
%      {Open-source Java library that provides tools to manipulate geospatial data.}    


% \item{2006 -- 2009}
%      {Terralib (INPE -- Spatial Research Institute) \newline {\footnotesize \href{http://www.terralib.org}{http://www.terralib.org}} \newline {\footnotesize \href{http://www.inpe.br}{http://www.inpe.br}}}
%      {Geographic Database written in C++.}

% \end{yearlist}

% \section{Selected Publications}

% \begin{enumerate}

%   \item \textbf{Edgard Marx}, Ciro Baron, Tommaso Soru, S\"oren Auer (2017). {T}ransparently {S}hifting {Q}uery {E}xecution   on {K}nowledge {G}raphs to the {E}dge.
%   International Conference of Semantic Computing.

%   \item Saedeeh Shekarpour, \textbf{Edgard Marx}, S\"oren Auer and Amit Sheth (2017). {{RQUERY}: {R}ewriting {T}ext {Q}ueries to {A}lleviate the {V}ocabulary {M}ismatch {P}roblem on {RDF} {K}nowledge {B}ases}. Proceedings of the Thirty-first AAAI Conference on Artificial Intelligence, San Francisco, California USA.
  
%   \item \textbf{Edgard Marx}, Konrad H{\"o}ffner, Saeedeh Shekarpour, Axel-Cyrille Ngonga Ngomo, Jens Lehmann and S{\"o}ren Auer (2016). Exploring Term Networks for Semantic Search over RDF Knowledge Graphs.
%   Metadata and Semantics Research: 10th International Conference, MTSR 2016, G{\"o}ttingen, Germany, November 22-25, 2016, Proceedings.
  
%   \item \textbf{Edgard Marx}, Amrapali Zaveri, Diego Moussallem and Sandro Rautenberg (2016). {DBtrends} : {E}xploring {Q}uery {L}ogs for {R}anking {RDF} {D}ata.
%   Proceedings of the 12th International Conference on Semantic Systems.
    
%   \item \textbf{Edgard Marx}, Amprapali Zaveri, Mofeed Mohammed,  Sandro Rautenberg, Jens Lehmann, Axel-Cyrille Ngonga Ngomo and Gong Cheng (2016). {DBtrends} : {P}ublishing and {B}enchmarking {RDF} {R}anking {F}unctions. 2nd International Workshop on Summarizing and Presenting Entities and Ontologies, co-located with the 13th Extended Semantic Web Conference.
  
%   \item \textbf{Edgard Marx} and Sandro Coelho (2016). {Answering {L}ive {Q}uestions from {H}eterogeneous {D}ata {S}ources}. 
%   Proceedings of the 25th Text Retrieval Conference (TREC 2016), Live QA Track, 15 November 2016, Montgomery County MD, USA.
  
%   \item Tommaso Soru, \textbf{Edgard Marx}, and  Axel-Cyrille {Ngonga Ngomo (2015). {ROCKER} -- A Refinement Operator for Key Discovery}.
%   Proceedings of the 24th International Conference on World Wide Web, WWW 2015.
  
%   \item Saeedeh Shekarpour, \textbf{Edgard Marx}, Axel-Cyrille Ngonga Ngomo and Sören Auer (2015). {SINA}: {S}emantic {I}nterpretation of {U}ser {Q}ueries for {Q}uestion {A}nswering on {I}nterlinked {D}ata. Proceedings of the 10th International Conference on Semantic Systems. Journal of Web Semantics: Science, Services and Agents on the World Wide Web.
  
%  \item \textbf{Edgard Marx}, Ricardo Usbeck, Axel-Cyrille {Ngomo Ngonga}, Konrad H{\"o}ffner, Jens Lehmann and S{\"o}ren Auer (2014). {T}owards an {O}pen {Q}uestion {A}nswering {A}rchitecture. Proceedings of the 10th International Conference on Semantic Systems.
  
%  \item \textbf{Edgard Marx}, Tommaso Soru, Saedeeh Shekarpour, S\"oren Auer, Axel-Cyrille {Ngonga Ngomo} and  Karin Breitman (2013). {T}owards an {E}fficient {RDF} {D}ataset {S}licing.
%  International Journal of Semantic Computing.
 
%  \item \textbf{Edgard Marx}, Percy E. Salas, Karin Breitman, José Viterbo Filho and  Marco Antonio Casanova (2012). RDB2RDF: A relational to RDF plug-in for Eclipse.
%  Software Practice and Experience.
 
% \end{enumerate}

% \newpage

% \section{Community Services}

% \begin{factlist}
% \item{Reviewer}{ICSC 2017, KESW 2015, AAAI 2015, LISC 2014, KESW 2014, LDOW 2014, ICWSM 2013}
% \item{PC Member}{CSCUBS 2016, QALD-6, NLIWoD 2015, JBMS Bio-IR 2015, AISI 2015, LOD Brasil 2014, CLEF 2014 Labs, NLIWoD 2014}
% \item{Chair}{WEB 2017}
% \item{Presentations}{Semantics 2016, MTRS 2016, ISWC 2015, Semantics 2014, ICSC 2013, IBM Palo Alto 2013}
% \end{factlist}

% \section{Supervisions}

% \begin{factlist}
% \item{2015}{Felix Rauchfuss: DBtrends API}
% \item{2013}{Andy Wermky: SINA API}
% \end{factlist}

% \section{Awards \& Honors}

% \begin{factlist}
% \item{2016}{Best paper at Ontobras 2016}
% \item{2013}{Selected among four best papers at ICSC 2013}
% \item{2011}{Selected among three best papers at Software Practice \& Experience 2011}
% \end{factlist}



% \section{Communication skills}

% \begin{factlist}
% \item{Portugues}{Native}
% \item{English}{Professional Working Proficiency}
% \item{Spanish}{Limited Working Proficiency}
% \item{German}{Elemental Proficiency}
% \end{factlist}

% \section{Skills \& Experience}

% \begin{factlist}

% \item{Good level}
%      {Software Engineer, Cloud Computing, Geographic Information Systems, Databases, Information Retrieval, HCI, Algorithms, Semantic Technologies, Mass Monitoring Systems, Software Project Management, NLP}

% \end{factlist}